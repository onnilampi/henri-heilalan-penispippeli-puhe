\documentclass[a4paper, 12pt, finnish]{report}
\usepackage[utf8]{inputenc}
\usepackage{amsfonts}
\usepackage{graphics}
\usepackage[finnish]{babel}
\usepackage{titlesec}
\titleformat{\chapter}
{\Large\bfseries}
{}            
{0pt}      
{\huge} 

\newcommand{\topic}{Penispippeli}
\usepackage{hyperref}
\hypersetup{pdfpagemode=UseNone, pdfstartview=FitH, colorlinks=true,urlcolor=red,linkcolor=blue,citecolor=black,pdftitle={Penispipeli},pdfauthor={Onni Lampi}}
\setlength{\parindent}{0mm}
\setlength{\emergencystretch}{15pt}
\newcommand*{\findate}{\the\day.\the\month.\the\year}

\begin{document}



\section*{\topic}
Valmistuneet, toverit!\\

Olemme kaikki käyneet pitkän tien.
Kukaan tässä salissa ei ole täällä noutamassa vain turhaa paperia, vaan kaikilla on tässä vaiheessa jotain suurta saavutusta vyön alla.
Tässä tilaisuudessa on paikalla kandiksi, DIksi, ja tohtoriksi valmistuvia.
Kaikki me olemme tehneet tekoja ja opintoja, joita suurin osa tästä kansakunnasta ei suorita!
Me olemme niitä, joilla on "the right stuff". Sitä ei voi yksin sanoin kuvailla, vaan se on jotain mitä vain minun vertaiseni ymmärtävät.\\

Muistan kun aloitin opintoni omassa ohjelmassani syksyllä 2012 ja kaikki tuntui vaikealta.
Onneksi korkeakoulu ja kilta nappasivat minusta kopin, ja patistivat minut opiskelemaan muun työn ohella.
Oheistoiminta on ja on ollut kivaa, mutta silti kaiken taustalla on toiminut oma saavuttaminen.
Mikään ei ole ollut siistimpää, kuin asioiden oppiminen töiden ja (turhien opintojen) ohessa oppiminen.
Niiden avulla olen todennut, ettei mikään oikea opintopolku ole absoluuttinen.\\

Oikea oppiminen on sitä, että ymmärtää asioiden olevan väärin.
Kaikkien kertoma ja toitottama "totuus" ei oikeasti olekkaan just niin, vaan asiat pitää itse ymmärtää ja tajuta.
Aalto-yliopisto on opettanut minulle sen, ettei asiaoita pidä käsittää itsestään selvyyksinä, vaan kaikki on *oikeasti* mietittäviä juttuja.\\

Aalto-yliopistossa olen oppinut kyseenalaistamaan asioita ja miettinään itse, miten asioita tulisi tehdä.
Yliopisto on opettanut minut ajattelemaan kriittisesti ja objektiivisesti.
Ilman Aallon tarjoamaa opetusta en välttämättä ymmärtäisi, miten asiat oikeasti tehdään.
On tärkeää tajuta, mitä eroa on käytännön tekemisellä ja käytännön laskemisella.
Toinen ei voi olla olemassa ilman toista, mutta osaamalla vain puolet ei osaa mitään.\\

\end{document}
